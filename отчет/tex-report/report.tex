%%%%%%%%%%%%%%%%%%%%%%%%%%%%%%%%%%%%%%%%%%%%%%%%%%%%%%%%%%%%%%%%%%%%%%%%%%%%%
\documentclass[a4paper, 14pt]{extarticle}
\usepackage[top=20mm, left=30mm, right=10mm, bottom=20mm]{geometry}
\usepackage[utf8]{inputenc}
\usepackage[T2A]{fontenc}
\usepackage[english,russian]{babel}
\usepackage{indentfirst}
\usepackage{graphicx, animate}
\usepackage{amsmath}
\usepackage{float}
\usepackage{setspace}
\usepackage{listings}
\usepackage{amsmath}
\usepackage{diagbox}
\usepackage{color}
\usepackage[unicode, pdftex]{hyperref}
\usepackage{minted}
\graphicspath{{pictures/}}
\DeclareGraphicsExtensions{.png,.jpg}
\renewcommand{\thesection}{\arabic{section}}
\renewcommand{\theenumi}{\arabic{enumi}}
\renewcommand{\labelenumii}{\arabic{enumi}.\arabic{enumii}.}
%%%%%%%%%%%%%%%%%%%%%%%%%%%%%%%%%%%%%%%%%%%%%%%%%%%%%%%%%%%%%%%%%%%%%%%%%%%

\begin{document}
\section*{Актуальность}
	В ходе эксплуатации электронных устройств регулярно возникает необходимость в настройке. На входы устройства принимают сигналы, форма которых задается напряжением. Для тестирования и отладки могут понадобиться как цифровые, так и аналоговые сигналы разной формы. Формирование аналоговых сигналов может обеспечить специализированное устройство --- генератор сигналов.
	
	Генератор сигналов --- это неотъемлемый инструмент для любого специалиста в области электроники. На сегодняшний день разрабатывается достаточно много генераторов сигналов, но не все генераторы, которые есть на рынке, обладают компактными размерами, лёгкостью транспортировки и доступностью в цене. 
	Ранее практически все лабораторные генераторы были аналоговыми и конструировались на различных схемах. К их достоинствам можно отнести простоту и надёжность, но у них есть существенные недостатки в виде меньшей стабильности и более тщательной настройке. Сейчас практически все генераторы, которые есть на рынке создаются на основе цифровых методов синтеза аналоговых сигналов, т. к. они стабильные и точные.  
	 Такого рода генераторы могут найти применение и в промышленности, но не всем пользователям требуются такие высокие характеристики. Разработанный в данной работе генератор претендует на применение в домашней лаборатории в качестве простого и функционального дешёвого генератора сигналов.
	
	Применением такого генератора может быть генерация сигналов разных форм, работа с аналоговыми системами для исследования влияния сигналов на них, изучение методов обработки сигнала или основ электроники. 

\section*{Методы генерации}
Можно выделить несколько ключевых методов цифровой генерации сигнала, а также их основные преимущества и недостатки.

Среди всех методов, наиболее выделяется **метод DDS** за его универсальность,
 гибкость и простоту реализации. Он позволяет создавать различные формы сигналов
 с высокой точностью и быстродействием, что делает его идеальным выбором для создания функционального генератора.



\section*{Выбор МК и инструментов}
	Были рассмотрены два популярных семейства микроконтроллеров AVR и STM32.
	
	Исходя из таблицы можно сделать вывод, что микроконтроллеры AVR применимы в малом спектре задач где скорость не так важна. В нашем же случае скорость работы микроконтроллера может сильно влиять на генерацию сигнала, а также требуется объём памяти для хранения отсчётов сигналов. В микроконтроллерах STM32 с частотой и объёмом памяти проблем нет и они имеют широкое применение. Серию же выберем F103xC за её характеристики. В связи с этим, а также доступностью отладочных плат будет применён микроконтроллер STM32F103RCT6.
	
	Попользовавшись обеими средами разработки и разными библиотеками, а также основываясь на достоинствах и недостатках была выбрана среда разработки PlatformIO в связке с библиотекой libopencm3.
	
\section*{Программа}
	Структурно устройство будет выглядеть следующим образом (рис. 2.8). Цифро-аналоговый преобразователь будет использоваться встроенный в микроконтроллер, а в качестве дисплея будет выступать OLED экран с разрешением 128 на 64 пикселя, работающий по интерфейсу I2C.

	Программа должна выполнять три действия:
	
	\begin{enumerate}
		\item Вывод отсчёта в ЦАП;
		\item Обработка кнопок;
		\item Вывод информации на дисплей.
	\end{enumerate}
	
	
	Для цифро-аналогового преобразователя и кнопок выделим два таймера общего назначения, а работа с дисплеем будет идти в главном цикле программы. Применив такой подход, удастся добиться синхронного выполнения программы. 
	
	Подпрограмма обработки кнопок находится в обработчике прерывания таймера номер 3. Таймер настроен на период 250 миллисекунд. Благодаря такой организации, решается проблема дребезга кнопок. Не приходится делать программную или аппаратную задержку для ожидания установки состояния кнопки.
	
\section*{Аппарат}
	По структурной схеме накидали электрическую. Собрали прототип.

\end{document}