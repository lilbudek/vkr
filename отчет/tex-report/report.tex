%%%%%%%%%%%%%%%%%%%%%%%%%%%%%%%%%%%%%%%%%%%%%%%%%%%%%%%%%%%%%%%%%%%%%%%%%%%%%
\documentclass[a4paper, 14pt]{extarticle}
\usepackage[top=20mm, left=30mm, right=10mm, bottom=20mm]{geometry}
\usepackage[utf8]{inputenc}
\usepackage[T2A]{fontenc}
\usepackage[english,russian]{babel}
\usepackage{indentfirst}
\usepackage{graphicx, animate}
\usepackage{amsmath}
\usepackage{float}
\usepackage{setspace}
\usepackage{listings}
\usepackage{amsmath}
\usepackage{diagbox}
\usepackage{color}
\usepackage[unicode, pdftex]{hyperref}
\graphicspath{{pictures/}}
\DeclareGraphicsExtensions{.png,.jpg}
\renewcommand{\thesection}{\arabic{section}}
\renewcommand{\theenumi}{\arabic{enumi}}
\renewcommand{\labelenumii}{\arabic{enumi}.\arabic{enumii}.}
%%%%%%%%%%%%%%%%%%%%%%%%%%%%%%%%%%%%%%%%%%%%%%%%%%%%%%%%%%%%%%%%%%%%%%%%%%%

\begin{document}
\section*{Актуальность}
	В ходе эксплуатации электронных устройств регулярно возникает необходимость в настройке. На входы устройства принимают сигналы, форма которых задается напряжением. Для тестирования и отладки могут понадобиться как цифровые, так и аналоговые сигналы разной формы. Формирование аналоговых сигналов может обеспечить специализированное устройство --- генератор сигналов.
	
	Генератор сигналов --- это неотъемлемый инструмент для любого специалиста в области электроники. На сегодняшний день разрабатывается достаточно много генераторов сигналов, но не все генераторы, которые есть на рынке, обладают компактными размерами, лёгкостью транспортировки и доступностью в цене. 
	
	\textbf{Цель и задачи}. Потребуется метод генерации, его программная реализация, микроконтроллер и какой-то макет.

\section*{Методы генерации}
	Можно выделить несколько ключевых методов цифровой генерации сигнала, а также их основные преимущества и недостатки.

	Метод аппроксимации подразумевает собой вычисление отсчётов
функции по заданным параметрам. В устройстве хранятся только параметры, определяющие генерируемый сигнал. 

	Для генерации сигналов также применяется итерационный метод CORDIC. Переводится как цифровое вычисление поворота системы координат. Он был разработан для аппаратного поворота вектора на плоскости. 
	
	В табличном методе генерации сигналов предполагается, что заранее
вычисленные отсчёты хранятся в памяти. То есть никаких вычислений не требуется и генерация сводится к тому, что в порт цифро-аналогового преобразователя нужно вывести ячейку, но имеет проблему с регулировкой частоты.

 	К табличным методам относится также метод прямого цифрового синтеза или как его ещё называют метод DDS и он решает проблему, в которую упирается обычный табличный метод.

	Среди всех методов, наиболее выделяется **метод DDS** за его универсальность,
 гибкость и простоту реализации. Он позволяет создавать различные формы сигналов
 с высокой точностью и быстродействием, что делает его идеальным выбором для создания функционального генератора.
 
\textbf{Метод DDS}. Код частоты задаёт выходную частоту генератора. Если выделить больше памяти для аккумулятора фазы, то регулировка частоты может стать точнее.

\section*{Существующие генераторы}	
	В настоящее время существуют полноразмерные генераторы сигналов и генераторы реализованные всего лишь на одной микросхеме. Полноразмерные обладают высокими характеристиками, но не у всех пользователей есть в них надобность и тем более стоимость за такие характеристики очень существенная, а также у полноразмерного генератора большие габариты.
	
	Быстрое и непрерывное развитие схемотехники привело к появлению маленьких микросхем, реализующих функционал генератора сигналов. Например, микросхема программируемого генератора AD9833. Её преимущества включают низкую стоимость, малые размеры, низкое энергопотребление и при этом высокое качество сигнала. Однако, требует управления через микроконтроллер и соответственно специальную программу для настройки параметров сигнала.

	\textit{Исходя из достоинств и недостатков}. Гораздо эффективнее будет разработать генератор сигналов на одном микроконтроллере тем самым будут сэкономлены ресурсы и система получится довольно гибкой.
\section*{Выбор МК и инструментов}
	Были рассмотрены два популярных семейства микроконтроллеров AVR и STM32.
	
	Исходя из таблицы можно сделать вывод, что микроконтроллеры AVR применимы в малом спектре задач где скорость не так важна. В нашем же случае скорость работы микроконтроллера может сильно влиять на генерацию сигнала, а также требуется объём памяти для хранения отсчётов сигналов. В микроконтроллерах STM32 с частотой и объёмом памяти проблем нет и они имеют широкое применение. Серию же выберем F103xC за её характеристики. В связи с этим, а также доступностью отладочных плат будет применён микроконтроллер STM32F103RCT6.
	
	Попользовавшись средами разработки и разными библиотеками, а также основываясь на достоинствах и недостатках была выбрана среда разработки PlatformIO в связке с библиотекой libopencm3.
	
\section*{Программа}
	Структурно устройство будет выглядеть следующим образом. Цифро-аналоговый преобразователь будет использоваться встроенный в микроконтроллер, а в качестве дисплея будет выступать OLED экран с разрешением 128 на 64 пикселя, работающий по интерфейсу I2C.

	Программа должна выполнять три действия.
	
	Для цифро-аналогового преобразователя и кнопок были выделены два таймера общего назначения, а работа с дисплеем организована в главном цикле программы. Применив такой подход, удастся добиться синхронного выполнения программы. 
	
	Подпрограмма вывода отсчёта в ЦАП содержит в себе вывод значения и вычисления адреса, в котором заложен метод DDS.
	
	Подпрограмма обработки кнопок находится в обработчике прерывания таймера номер 3. Таймер настроен на период 250 миллисекунд. Благодаря такой организации, решается проблема дребезга кнопок. Не приходится делать программную или аппаратную задержку для ожидания установки состояния кнопки. Состоит функция из 5 подпрограмм (пояснить каких).
	
	В главном цикле помимо работы с дисплеем расположены настройки периферии (что за периферия).
	
\section*{Аппарат}
	По структурной схеме был создан фрагмент схемы электрической принципиальной. Микроконтроллер с минимальной обвязкой и периферия.
	
	Сконструирована плата периферии. 
	 
	Затем получился макет устройства, состоящий из отладочной платы и платы периферии.
	
\section*{Тестирование}
	После создания макета была проверена его работоспособность. На слайде можно видеть состояние самого устройства и что происходит в отладчике.
	
	На осциллографе были сняты разные формы сигнала. Здесь представлен синусоидальный сигнал с частотой 1875 Гц, то есть то что мы выставляли на устройстве. 
	
	Было проведено моделирование в Octave. Были взяты записанные осциллографом отсчёты синусоидального сигнала на частоте 1 кГц и построена для сравнения такая же форма сигнала, но с заведомо большей частотой дискретизации. Считаем этот сигнал идеальным. В итоге получим следующее изображение, из которого можно заметить, что сигнал с генератора не совпадает с идеальным, то есть его частота не составляет ровно 1 кГц. Ввиду того, что источник тактирования не совсем точный.
	
	Проведён спектральный анализ нашего сигнала, перейдя от временной составляющей к частотной с помощью быстрого преобразования Фурье. Можно заметить утечку фазы.
	
	\textit{Большой столбец --- постоянная составляющая, так как сигнал однополярный. Главная гармоника на частоте 1 кГц.}

\end{document}