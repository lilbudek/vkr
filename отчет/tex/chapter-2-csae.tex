\chapter{ПРОЕКТИРОВАНИЕ}
\section{Моделирование DDS}
Для начала потребуется таблица отсчётов, чтобы её вычислить используем готовый инструмент.
	
	\begin{figure}[H]
    \centering
    \includegraphics[width=1\textwidth]{../image/lut_prog.png}
    \caption{Программа для вычисления значений сигнала.}
	\end{figure}
	
	У таблицы есть 4 параметра:
	\begin{enumerate}
		\item Разрядность ЦАП: 8 или 12 бит.
		\item Максимальное значение.
		\item Количество значений.
		\item Смещение от нуля.
	\end{enumerate}
	
	Использовать мы будем 12-битные значения в количестве 256 чисел. Максимальное значение амплитуды сигнала может быть 4095.%, но так как для улучшения генерации будет задействован встроенный в цифро-аналоговый преобразователь выходной буфер, то он будет срезать сигнал сверху и снизу на 0.2В, поэтому значения тоже следует срезать на эту же величину для корректной генерации.
	
%	В документе от ST про работу с цифро-аналоговым преобразователем есть формула для расчета выходного напряжения.
	
%	$DAC_{output} = V_{REF}*\dfrac{DOR}{DAC_{MaxDigitalValue} + 1}$, где DOR --- цифровое значение.
	
%	Нам нужно найти какое значение соответствует напряжению 0.2В. Выразим DOR и подставим имеющиеся значения.
	
%	$DOR = \dfrac{V_{REF}}{DOR}*DAC_{MaxDigitalValue} + 1 = \dfrac{3.3}{0.2}*(4095+1) = 248$
	
%	Укажем смещение от нуля 248, а максимальное значение 4095 меньше на 248, то есть 3847 и сгенеририуем таблицу отсчётов для синусоиды. 
	
	\begin{figure}[H]
    \centering
    \includegraphics[width=1\textwidth]{../image/lut.png}
    \caption{Вычисление таблицы сигнала.}
	\end{figure}
	
	Теперь у нас есть данные для генерации сигнала. Смоделируем алгоритм метода прямого цифрового синтеза на языке Си для дальнейшей реализации на микроконтроллере.

\begin{code}
\captionof{listing}{Метод DDS.}
\begin{minted}[mathescape,linenos,frame=lines,breaklines]{text}
int main() {
  uint16_t p_acc, p_step;
  uint8_t addr = 0; // адрес ячейки

  p_acc = 0;    // аккумулятор фазы
  p_step = 128; // код частоты

  while(1)
  {
    addr = p_acc >> 8; // выделение старшей части аккумулятора фазы
    p_acc += p_step;   // шаг
    printf("%d 0x%X\n", addr, sinus[addr]); // вывод отсчёта
  }

  return 0;
}
\end{minted}
\end{code}

	
	Код частоты задаёт выходную частоту генератора. При значении 256 вывод будет следующий:
	
\begin{figure}[H]
    \centering
    \includegraphics[width=0.6\textwidth]{../image/dds256.png}
    \caption{Формирование отсчётов при коде частоты 256.}
\end{figure}
	
	Увеличим код частоты в два раза и получим следующее:

\begin{figure}[H]
    \centering
    \includegraphics[width=0.6\textwidth]{../image/dds512.png}
    \caption{Формирование отсчётов при коде частоты 512.}
\end{figure}

	Как можно заметить отсчёты стали формироваться через один, соответственно частота вырастит в два раза. Теперь уменьшим частоту в два раза выставив код частоты 128.

\begin{figure}[H]
    \centering
    \includegraphics[width=0.6\textwidth]{../image/dds128.png}
    \caption{Формирование отсчётов при коде частоты 128.}
\end{figure}

	Программа стала выводить каждый отсчёт по два раза тем самым, понизив частоту.
	
	В данном виде модуляции код частоты просто абстрактное число, которое добавляется к аккумулятору фазы и узнать реальную частоту проблематично. Результат синтеза будет проверен опытным путём на микроконтроллере.
	
\section{Обзор микроконтроллеров}

\section{Среды разработки для STM32}

\subsection{STM32CubeIDE}
	STM32CubeIDE --- это продвинутая платформа разработки на C/C++ с функциями настройки периферийных устройств, генерации кода, компиляции кода и отладки для микроконтроллеров и микропроцессоров STM32~\cite{cube}. Среда разработки основана на платформе Eclipse и GCC toolchain для разработки и GDB для отладки. Она позволяет интегрировать сотни существующих плагинов, которые дополняют возможности Eclipse IDE. Имеет расширенные функции отладки, включая: просмотр ядра ЦП, регистров периферийных устройств и памяти, анализ системы просмотра переменных в режиме реального времени. Поддерживается на операционных системах: Linux, macOS, Windows.

	\begin{figure}[H]
    \centering
    \includegraphics[width=0.85\textwidth]{../image/cube.jpg}
    \caption{Интерфейс STM32CubeIDE.}
	\end{figure}
	
	После выбора микроконтроллера STM32 создается проект и генерируется код инициализации. В любой момент разработки пользователь может вернуться к инициализации и настройке периферийных устройств и повторно создать код инициализации без какого-либо влияния на пользовательский код. Для разработки используется библиотека HAL. 
	
	Драйверы HAL включают в себя полный набор готовых к использованию функций, которые упрощают реализацию пользовательских приложений. Например, коммуникационные периферийные устройства содержат функции для инициализации и настройки периферийного устройства, управления передачей данных, обработки прерываний или DMA~\cite{hal}.
	
	Достоинства:
	\begin{itemize}
		\item Поддержка различных ОС.
		\item Расширенные возможности отладки.
		\item Большое сообщество.
		\item Автогенерация кода.
	\end{itemize}
	
	Недостатки:
	\begin{itemize}
		\item Требовательность к ресурсам ПК.
		\item Сложность настройки.
	\end{itemize}

\subsection{PlatformIO}

	PlatformIO --- удобная и расширяемая интегрированная среда разработки с набором профессиональных инструментов разработки, предоставляющая современные и мощные функции для ускорения и упрощения процесса разработки встраиваемых продуктов~\cite{plio}.
	
	\begin{figure}[H]
    \centering
    \includegraphics[width=0.85\textwidth]{../image/plio.jpg}
    \caption{Интерфейс PlatformIO.}
	\end{figure}
	
	Данная среда разработки является расширением для текстового редактора Visual Studio Code. VS Code --- это легкий, но мощный редактор кода, имеющий богатую экосистему расширений~\cite{docsplio}. Доступен для Windows, macOS и Linux. Работа в паре с VS Code позволяет удобно форматировать код и пользоваться расширениями для языков программирования.
	
	PlatformIO позволяет работать со многими микроконтроллерами и поддерживает множество фреймворков для них, а также библиотек. Ввиду такой широкой поддержки, для STM32 можно разрабатывать с удобной для себя библиотекой. Это может быть к примеру тот же HAL, что и в STM32CubeIDE или libopencm3. Проект libopencm3 (ранее известный как libopenstm32) направлен на создание бесплатной библиотеки микропрограмм с открытым исходным кодом (LGPL версии 3 или более поздней) для различных микроконтроллеров ARM Cortex-M3, включая ST STM32.%, Toshiba TX03, Atmel SAM3U, NXP LPC1000, EFM32 и других ~\cite{lcm3}.


	Достоинства:
	\begin{itemize}
		\item Поддержка различных ОС.
		\item Быстрая компиляция.
		\item Поддержка GitHub.
		\item Возможность работать с разными фрэймворками и платформами.
	\end{itemize}
	
	Недостатки:
	\begin{itemize}
		\item Высокий порог вхождения.
		\item Сложность установки.
	\end{itemize}

\subsection{IAR Embedded Workbench}




\section{Алгоритм работы}

\section{Разработка схемы}
\begin{figure}[H]
    \centering
    \includegraphics[width=1\textwidth]{../image/scheme-cropped.pdf}
    \caption{Схема электрическая принципиальная.}
\end{figure}

\section{Вывод из второй главы}
