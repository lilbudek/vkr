\chapter{Начало разработки}

\section{Обзор микроконтроллеров}



\section{Среды разработки для STM32}




\section{Методы программной генерации сигнала}
	Основные методы цифровой генерации сигналов --- метод аппроксимации и табличный метод.
	
	Метод аппроксимации подразумевает собой вычисление отсчётов функции с заданным интервалом. В памяти хранятся только параметры сигнала. Поэтому данный метод позволяет затратить небольшой объём памяти, но его недостаток это затраты на вычисления, что ограничивает максимальную частоту сигнала.
	
	В табличном методе генерации сигналов предполагается, что заранее вычисленные отсчёты хранятся в памяти. То есть никаких вычислений не требуется и генерация сводится к тому, что в порт цифро-аналогового преобразователя нужно вывести ячейку по заданному адресу. Таким образом, время на формирование отсчёта становится меньше и появляется возможность генерировать сигнал с более высокой частотой. Недостатком же является большие затраты памяти.
	
	Будем рассматривать табличный метод синтеза. Для начала потребуется таблица отсчётов. Её параметрами являются количество отсчётов и амплитуда.