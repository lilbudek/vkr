\chapter{название}

\section{Первый раздел главы 1}

Пример использования minted для оформления кода.
\begin{code}
\captionof{listing}{Сложение двух массивов параллельно десятью потоками (пример из https://ru.wikipedia.org/wiki/OpenMP)}
\label{code:pi-example}
\begin{minted}[mathescape,linenos,frame=lines,breaklines]{C}
#include <stdio.h>
}
\end{minted}
\end{code}

\section{Второй раздел главы 1}
Ниже представлен многоуровневый список:
\begin{enumerate}
 \item 1
 \begin{enumerate}
 \item 2
 \item 3
\end{enumerate}
\item 4
 \begin{enumerate}
 \item 5
  \begin{enumerate}
 \item 6
 \item 7
\end{enumerate}
 \item 8
\end{enumerate}
\item 9
\end{enumerate}

\subsection{Пример подраздела}
Текст из первого подразделя для проверики отступа между абзацами. Текст из первого подразделя для проверики отступа между абзацами. Текст из первого подразделя для проверики отступа между абзацами.

Текст из первого подразделя для проверики отступа между абзацами. Текст из первого подразделя для проверики отступа между абзацами. Текст из первого подразделя для проверики отступа между абзацами.

\begin{table}[H]
\caption{\label{t1}Системные требования}
\begin{tabular}{|p{3 cm}|p{3 cm}|p{3 cm}|p{5 cm}|}
\hline
Минимальные требования & 1 & 2 & 3 \\ \hline
Версия операционной системы & 1 & 2 & 3 \\ \hline
Процессор & 1 & 2 & 3 \\ \hline
Графический API & 1 & 2 & 3 \\ \hline
\end{tabular}
\end{table}

Пример ссылки на рисунок в документе~\ref{fig:example01}.
\begin{figure}[h]
    \centering
    %\includegraphics[width=1.0\textwidth]{../images/screen.png}
    \caption{Пример рисунка}
    \label{fig:example01}
\end{figure}

Пример ссылки на рисунок в документе~\ref{fig:example04}.
\begin{figure}[h]
    \centering
    %\includegraphics[width=1.0\textwidth]{../images/screen2.png}
    \caption{Пример рисунка}
    \label{fig:example04}
\end{figure}
