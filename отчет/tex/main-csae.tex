\documentclass[14pt, oneside]{altsu-bachelor}

\title{Программный генератор сигналов на основе микроконтроллера STM32}
\author{Д.\,С.~Вебер}
\groupnumber{506}
\GradebookNumber{294}
\supervisor{П.\,Н.~Уланов}
\supervisordegree{ст. преп.}
\ministry{Министерство науки и высшего образования}
\country{Российской Федерации}
\fulluniversityname{ФГБОУ ВО Алтайский государственный университет}
\institute{Институт цифровых технологий, электроники и физики}
\department{Кафедра вычислительной техники и электроники}
\departmentchief{В.\,В.~Пашнев}
\departmentchiefdegree{к.ф.-м.н., доцент}
\shortdepartment{ВТиЭ}
\ChairmanOfTheStateCertificationCommission{С.\,П.~Пронин}
\ChairmanOfTheStateCertificationCommissiondegree{д.т.н., проф.}
\NormController{А.\,В.~Калачёв}
\NormControllerdegree{к.ф.-м.н., доцент}
\Consultant{}
\Consultantdegree{}
\UDC{519.688}
\docname{БР 09.03.01}
\abstractRU
{
	Данная выпускная квалификационная работа посвящена разработке программного генератора сигналов. В рамках работы были исследованы теоретические основы и методологии генерации сигналов, проведён обзор аналогичных генераторов сигналов, рассмотрены микроконтроллеры для реализации.
	
	Разработан макет программного генератора сигналов. Генератор способен генерировать различные формы сигналов (синусоидальная, треугольная, прямоугольная, пилообразная, обратно пилообразная) с частотой от 125 до 50 000 Гц, амплитудой 3 В и шагом по частоте 125, 250, 500, 1000 Гц. Для управления генератором используются пять кнопок, а информация о параметрах сигнала отображается на экране с разрешением 128x64 пикселя через интерфейс I2C.
}

\keysRU{генератор сигналов, микроконтроллер, цифро-аналоговый преобразователь, прямой цифровой синтез, макет}

\date{\the\year}

% Подключение файлов с библиотекой.
\addbibresource{graduate-students.bib}

\begin{document}
\maketitle

\setcounter{page}{2}
\makeabstract
\tableofcontents

\chapter*{ВВЕДЕНИЕ}
\addcontentsline{toc}{chapter}{Введение}
%нужно зайти из далека для настройки калибровки радио 
	В ходе эксплуатации электронных устройств регулярно возникает необходимость в настройке. На входы устройства принимают сигналы, форма которых задается напряжением. Для тестирования и отладки могут понадобиться как цифровые, так и аналоговые сигналы разной формы. Формирование аналоговых сигналов может обеспечить специализированное устройство --- генератор сигналов.
	
%	Для калибровки и отладки приборов необходимы колебания разных форм и периодов. Формирование требуемых электрических колебаний может обеспечить специализированное устройство --- генератор сигналов.
	
	Генератор сигналов --- это неотъемлемый инструмент для любого специалиста в области электроники. %У многих специалистов есть необходимость без применения дорогостоящих средств генерировать колебания с установленными параметрами. 
	На сегодняшний день разрабатывается достаточно много генераторов сигналов, % которые используются для различных исследований или необходимы для настройки каких-либо устройств,
	 но не все генераторы, которые есть на рынке, обладают компактными размерами, лёгкостью транспортировки и доступностью в цене. %Существует два основных вида генераторов --- аналоговый и цифровой.%
	
	Ранее практически все лабораторные генераторы были аналоговыми и конструировались на различных схемах. К их достоинствам можно отнести простоту и надёжность, но у них есть существенные недостатки в виде меньшей стабильности и более тщательной настройке. Сейчас практически все генераторы, которые есть на рынке создаются на основе цифровых методов синтеза аналоговых сигналов, %Цифровые генераторы легко интегрируются с другими системами и могут управляться через программное обеспечение, что упрощает процесс.%
	 т. к. они стабильные и точные. %., а также могут быть реализованы просто на микроконтроллере. 
	 Такого рода генераторы могут найти применение и в промышленности, но не всем пользователям требуются такие высокие характеристики. Разработанный в данной работе генератор претендует на применение в домашней лаборатории в качестве простого и функционального дешёвого генератора сигналов.
	
	Применением такого генератора может быть генерация сигналов разных форм, работа с аналоговыми системами для исследования влияния сигналов на них, изучение методов обработки сигнала или основ электроники. 
	
\textbf{Цель}
	выпускной квалификационной работы состоит в разработке программного генератора сигналов на микроконтроллере.

\textbf{Задачи}

	\begin{enumerate}
		\item Исследовать методы генерации сигналов и осуществить выбор;
		\item Рассмотреть семейства микроконтроллеров и осуществить выбор;
		\item Выбрать среду разработки;
		\item Разработать программу;
		\item Спроектировать устройство;
		\item Протестировать генератор.
	\end{enumerate}

\chapter{Теоретическая часть}

\section{Развитие генераторов сигналов}
	История развития генераторов сигналов начинается с аналоговых устройств, которые
использовались для генерации различных форм сигналов, включая низкочастотные,
высокочастотные, сверхвысокочастотные и импульсные. <<Во времена СССР потребности в новых средствах генерации сигналов удовлетворялись разработкой огромного числа всевозможных аналоговых генераторов сигналов>>~\cite{dgs}. Однако, с развитием технологий и потребностями в более сложных и модулируемых сигналах, стало очевидно необходимость в универсальных генераторах сигналов, способных генерировать сигналы типовых форм, такие как синусоидальные, прямоугольные, пилообразные и треугольные.

	В результате развития технологий и потребностей в более сложных и модулируемых
сигналах, появились новейшие разработки генераторов сигналов на основе прямого
цифрового синтеза частот и форм сигналов. Эти генераторы сигналов используют минимальное количество аналоговой элементной базы и основываются на стандартных и
специализированных сверхскоростных цифровых микросхемах, а также аналого-цифровых (АЦП) и цифро-аналоговых (ЦАП) преобразователях. Это позволяет легко интегрировать такие генераторы с цифровыми системами и современными компьютерами, открывая широкие возможности их применения в испытании и отладке различных электронных и радиотехнических систем и устройств. 

	В современной измерительной технике генераторы сигналов играют ключевую роль, особенно в области электронно-оптических приборов, видеоимпульсных и ультразвуковых локаторов, гео- и подповерхностных радаров, а также в системах цифровой связи, включая мобильные системы. Несмотря на то, что в прошлом развитие в этой области было активно, в настоящее время наблюдается отставание от многих передовых направлений применения электронных устройств, включая микропроцессоры, работающие на частотах в единицы ГГц и выше. 

%Важно отметить, что развитие генераторов сигналов тесно связано с развитием полупроводниковой технологии  элементной базы. В частности, были проведены значительные исследования в области германиевых и кремниевых транзисторов в лавинном режиме работы, что позволило разработать уникальные импульсные устройства и генераторы мощных импульсов. Однако, после распада СССР, многие из этих разработок были прерваны, и на рынок начали поступать зарубежные разработки. 
В целом, история развития генераторов сигналов отражает эволюцию технологий, потребностей в модулируемых сигналах и влияние глобальных изменений в науке и технике.

\section{Основные типы сигналов}
	Для начала стоит дать определение, что такое сигнал. <<Термин "сигнал" происходит от слова signum (знак), где знак подразумевается не в смысле полярности напряжения или тока, а в информационном смысле — сигналы являются переносчиками неких знаков, образующих информационную основу (алфавит) передаваемых сообщений>>~\cite{dgs}. Таким образом, можно сделать вывод, что постоянное напряжение $U=const$ и постоянный ток $I=const$ сигналами не являются, т.к. их параметры во времени не меняются.

	Впрочем постоянный ток или напряжение можно отнести к простейшим сигналам, которые несут в себе информацию о полярности напряжения или тока и их величинах, но в основном они не используются в качестве сигналов, а задают смещение чаще всего синусоидальным сигналам.

	Рассмотрим некоторые распространённые типы сигналов (напряжений, которые определённым образом меняются во времени).
\subsection*{Синусоидальные сигналы}
	Именно синусоидальные сигналы мы извлекаем из розетки. Математическое выражение, описывающее синусоидальное напряжение, имеет вид:

	$U=A sin2 \pi ft$, где $А$ --- амплитуда сигнала, а $f$ --- частота в герцах.

	\begin{figure}[H]
    \centering
    \includegraphics[width=0.575\textwidth]{../image/s_sin.png}
    \caption{Синусоидальный сигнал.}
	\end{figure}
	Эффективное значение равняется двойной амплитуде, то есть размаху сигнала. 

	Если нужно переместить начало координат ($t=0$) в какой-то момент времени, то в формулу следует добавить фазу:

	$U=A sin2 \pi ft + \theta$

	Синусоидальные сигналы характеризуются тремя параметрами:
	\begin{itemize}
		\item $U_{M}$ или $I_{M}$ --- амплитуда переменного напряжения или тока;
		\item $f$ --- частота (период);
		\item $\theta$ --- фазовый сдвиг.
	\end{itemize}

	Данный тип сигналов является периодическим, т. е. временная зависимость повторяется и есть условия:

	$u(t)=u(t+T)$

	$i(t)=i(t+T)$

	где $T=\frac{1}{f}$ --- период повторения сигнала.

	<<Основное достоинство синусоидальной функции (а также основная причина столь широкого распространения синусоидальных сигналов) состоит в том, что эта функция является решением целого ряда линейных дифференциальных уравнений, описывающих как физические явления, так и свойства линейных цепей.>>~\cite{is1}. Если подать на вход линейной цепи синусоидальный сигнал, то на выходе мы также получим синусоиду, но как правило с другой амплитудой и фазой. На практике поведение схемы оценивают по её амплитудно-частотной характеристике (АЧХ), которая показывает, как изменяется амплитуда синусоидального сигнала в зависимости от частоты. Для примера на усилителе звуковых частот амплитудно-частотная характеристика в идеале имеет ровную линию в диапазоне от 20 Гц до 20 кГц. Чаще всего частоты, с которыми приходится работать на синусоидальном сигнале, лежат в диапазоне от нескольких герц до нескольких мегагерц.

\subsection*{Линейно-меняющийся сигнал}
	Линейно-меняющийся сигнал --- это напряжение, возрастающее (или убывающее) с постоянной скоростью.

\begin{figure}[H]
     \begin{subfigure}[H]{0.45\textwidth}
         \centering
         \includegraphics[width=0.70\textwidth]{../image/s_la.png}
         \caption{Возрастающее напряжение в виде сигнала.}
     \end{subfigure}
     \hfill
     \begin{subfigure}[H]{0.45\textwidth}
         \centering
         \includegraphics[width=0.70\textwidth]{../image/s_lb.png}
         \caption{Ограниченный сигнал.}
     \end{subfigure}
        \caption{Линейно-меняющийся сигнал.}
\end{figure}

	Напряжение не может, конечно, расти бесконечно. Поэтому обычно данная величина имеет конечное значение (рис. 1.2 (б)) или сигнал становиться пилообразным (рис. 1.3).

	\begin{figure}[H]
    \centering
    \includegraphics[width=0.45\textwidth]{../image/s_saw.png}
    \caption{Пилообразный сигнал.}
	\end{figure}

\subsection*{Треугольный сигнал}
	Треугольный сигнал очень похож на линейно-меняющийся, но его отличие в том, что он симметричный.

	\begin{figure}[H]
    \centering
    \includegraphics[width=0.45\textwidth]{../image/s_tri.png}
    \caption{Треугольный сигнал.}
	\end{figure}

\subsection*{Шум}
	
	Также существует потребность в генерации шумов для анализа реакции на такой сигнал схемы. Характеризуется частотным спектром (произведение мощности на частоту в герцах). Самое распространённое шумовое напряжение --- белый шум с распределением Гаусса.  

	\begin{figure}[H]
    \centering
    \includegraphics[width=0.45\textwidth]{../image/s_noise.png}
    \caption{Сигнал шума.}
	\end{figure}

\subsection*{Прямоугольный сигнал}
	Прямоугольный сигнал или как его ещё называют меандр, характеризуется так же как и синусоидальный сигнал частотой и амплитудой.
	\begin{figure}[H]
    \centering
    \includegraphics[width=0.6\textwidth]{../image/s_p.png}
    \caption{Прямоугольный сигнал.}
	\end{figure}

	Если на вход линейной схемы подать прямоугольный сигнал, то на выходе вряд ли будет прямоугольник. Эффективным значением для данного сигнала является значение его амплитуды. <<Форма реального прямоугольного сигнала отличается от идеального прямоугольника; обычно в электронной схеме время нарастания сигнала $t_{H}$ составляет от нескольких наносекунд до нескольких микросекунд.>>~\cite{is1}. 

	\begin{figure}[H]
    \centering
    \includegraphics[width=0.7\textwidth]{../image/s_p_t.png}
    \caption{Время нарастания скачка прямоугольного сигнала.}
	\end{figure}
	
	На рисунке 1.7 изображено как обычно выглядит скачок сигнала прямоугольника. Время когда сигнал нарастет определяется в промежутке от 10 до 90\% максимальной амплитуды сигнала.

\subsection*{Импульсы}
Сигналы в виде импульса изображены на рисунке 1.8.

	\begin{figure}[H]
    \centering
    \includegraphics[width=0.65\textwidth]{../image/s_i.png}
    \caption{Импульсы.}
	\end{figure}
	
	Данный вид сигналов характеризуется амплитудой и длительностью импульса. Можно генерировать последовательность периодических импульсов и тогда можно ещё характеризовать сигнал частотой (повторением импульса). У импульсов есть полярность --- положительная и отрицательная. Кроме этого импульс может спадать, а может нарастать. 
	


\subsection*{Скачки и пики}
	Часто можно слышать о сигналах в виде скачков и пиков, но на самом деле широкого применения они не находят. <<К их помощи прибегают для описания работы схем>>~\cite{is1}. Данный вид сигналов изображён на рисунке 1.9.

	\begin{figure}[H]
    \centering
    \includegraphics[width=0.6\textwidth]{../image/s_sp.png}
    \caption{Сигнал в виде скачка и пика.}
	\end{figure}

	Скачок представляет из себя отдельную часть прямоугольного сигнала, в то время как пик представляет собой два скачка, разделенных очень коротким промежутком.


\section{Виды генераторов}
	Источник сигнала часто является неотъемлемой частью схемы, но для тестирования работы удобно иметь отдельный, независимый источник сигнала. В качестве такого источника могут использоваться следующие виды генераторов.
\begin{enumerate}
	\item Генераторы синусоидальных сигналов.
	\item Функциональные генераторы.
	\item Генераторы сигналов произвольной формы.
	\item Генераторы импульсов.
\end{enumerate}

\subsection*{Генераторы синусоидальных сигналов}
	Генераторы таких сигналов широко применяются при тестировании различных радиоэлектронных устройств. <<Достоинством обычных генераторов синусоидальных сигналов является возможность получения синусоидальной формы выходного сигнала с малыми нелинейными искажениями. А главным недостатком — низкая стабильность частоты.>>~\cite{dgs}. Сами же синусоидальные сигналы являются простейшими. Они изменяются во времени, но их параметры --- амплитуда, частота и фаза остаются постоянными. Изменяя эти параметры, возможно осуществить модуляцию синусоидальных сигналов и использовать их для переноса информации. На таком принципе построены разнообразные области применения синусоидальных сигналов в технике электросвязи и радиотехнике.

	В области измерительных приборов существуют различные виды генераторов синусоидального напряжения:

\begin{enumerate}
	\item Высокочастотные LC-генераторы.
	\item Низкочастотные RC-генераторы.
	\item Генераторы с разными типами резонаторов (кварцевые, пьезоэлектрические).
	\item Генераторы, которые формируют синусоиды, плавно ограничивая сигнал треугольника.
	\item Генераторы построенные на основе цифровых методах синтеза синусоидального сигнала.
\end{enumerate}

	Конечно в настоящее время первые четыре типа генераторов уже прошлый век. Развитие цифровых и вычислительных технологий способствовало созданию и широкому распространению генераторов пятого типа, которые используют цифровые методы для генерации синусоидальных и различных других форм сигналов.

\subsection*{Функциональные генераторы}
	Функциональными генераторами обычно называют генераторы, которые могут создавать несколько функциональных зависимостей. Данные устройства генерируют сигналы разной формы. Их простота и плавная регулировка частоты в большом диапазоне привела к массовому применению генераторов такого типа. Из всех генераторов, генераторы функций являются очень гибкими. Они позволяют генерировать синусоидальные, треугольные и прямоугольные сигналы в широком спектре частот, при этом возможно регулировать амплитуду и смещать сигнал по постоянному току. Благодаря такому разнообразию сигналов, сфера применения таких генераторов сильно расширяется. Данный вид источника сигнала может быть одним на все случаи жизни. Их можно использовать для тестирования, исследования и отладки абсолютно разной электронной аппаратуры. <<Наиболее часто функциональные генераторы используются при отладке ВЧ, НЧ и сверхнизкочастотных устройств. В СВЧ диапазоне частот эти устройства не используются, за исключением применения в качестве источников модулирующих сигналов.>>~\cite{dgs}.

	Функциональные генераторы также существуют как аналоговые так и цифровые, но в настоящее время аналоговые неактуальны. Переход к к функциональным генераторам с цифровым синтезов выходных сигналов и цифровой элементной базой связан с растущими требованиями к сигналам источника. У сигнала должна быть стабильная частота с амплитудой и верная форма. Благодаря применению цифровых элементов в массовой продукции (персональный компьютер, мобильный телефон), цифровые интегральные схемы стали бурно развиваться. Стала повышаться функциональность схем и понижаться их стоимость.



\subsection*{Генераторы сигналов произвольной формы}
	Данный вид генератора дополняет функциональный генератор. Достаточно новое направление в генераторах сигналов, которое основывается на прямом цифровом синтезе различных сигналов, по сути произвольных форм. Прямой цифровой синтез открыл возможность построить новую группу цифровых генераторов сигналов --- как обилие стандартных функций, так и произвольных форм. Однако синтез сигналов произвольных форм неминуемо усложняет устройство, <<так как требует применения перепрограммируемой электрическим способом памяти, введения редактора форм сигналов и средств отображения синтезируемой формы сигнала.>>~\cite{dgs}. Следовательно, генераторы такого типа относятся к достаточно сложным и дорогим приборам.

	И всё же в ряде случаев данный вид генератора сигналов бывает очень необходим. С ростом сложности связной, телекоммуникационной, телевизионной и радиолокационной техники, увеличивается разнообразие форм сигналов, требующих тестирования.


\subsection*{Генераторы импульсов}
	Важно иногда передавать значительное количество энергии за короткий промежуток времени. Генерация импульсов необходима для тестирования и отладки импульсных систем. Это может быть радиолокатор или устройства и цифровые системы различного назначения. В радиолокации импульс направляется в пространство затем отражается от достигнутой цели и воспринимается радиолокационным приёмником. Получив информацию о времени задержки отражённого сигнала, можно оценить расстояние до цели, а проанализировав отражённый импульс можно сделать какие-то выводы о характере цели. Такого рода генераторы находят большое применение в качестве источников несинусоидальных сигналов. <<Импульсные сигналы нужны и в целом ряде других применений, например для запуска мощных лазерных диодов, построения ультразвуковых и видеоимпульсных локаторов, запуска ядерных и термоядерных процессов и даже при испытании многих электронных устройств, использующих импульсные сигналы или отдельные их свойств.>>~\cite{dgs}.


\section{Методы цифровой генерации сигнала}

	После рассмотрения видов генераторов сигналов можно сделать вывод о том, что способы получения сигнала также делятся на аналоговые и цифровые. Однако, в настоящее время аналоговые генераторы неактуальны и изучать способы генерации и схемы на аналоговой элементной базе большого смысла не имеет. Следует провести исследование цифровых методов генерации сигнала. 
\subsection*{Метод аппроксимации}	
	Метод аппроксимации подразумевает собой вычисление отсчётов функции по заданным параметрам. <<В памяти устройства хранятся лишь параметры генерируемого сигнала. Программа вычисляет отсчеты функции с некоторым заданным интервалом.>>~\cite{leso}. Исходя из этого, данный метод позволяет затратить небольшой объём памяти, но его недостаток это затраты на вычисления, что ограничивает максимальную частоту сигнала. 
	Одним из видов аппроксимации является ступенчатая. Ступенчатая аппроксимация заключает в себе замену гармонического колебания напряжением ступенчатой формы, которая будет мало отличаться от синусоидальной кривой. <<При ступенчатой аппроксимации аппроксимируемое гармоническое напряжение $u(t) = U_{m} sin \omega t$ дискретизируется по времени (равномерная дискретизация с шагом $\Delta t$), и в интервале, разделяющем два соседних момента времени $t_{i}$ и $t_{i+1}$, заменяют синусоидальное колебание напряжением постоянного тока --- ступенькой, высота которой равна значению аппроксимируемого напряжения в момент $t_{i}$, т. е. $u(t_{i}) = U_{m} sin \omega t_{i}$.>>~\cite{metr}. В результате замены получим ступенчатую линию вместо кривой. Число ступенек при заданном периоде определяется шагом дискретизации $p=\dfrac{T}{\Delta t}$.
	
	\begin{figure}[H]
    \centering
    \includegraphics[width=0.5\textwidth]{../image/apr.png}
    \caption{Ступенчатая аппроксимация.}
	\end{figure}
	
\subsection*{CORDIC}
	Следующий метод тоже предполагает вычисление отсчётов. Для генерации сигналов также применяется итерационный метод CORDIC. <<CORDIC --- это аббревиатура от Coordinate Rotation Digital Computer: цифровое вычисление поворота системы координат. Алгоритм "цифра за цифрой" был разработан для аппаратного поворота вектора на плоскости с помощью простых операций "сдвиг регистра вправо" и сложение/вычитание регистров.>>~\cite{cordic}. Смысл итерационного метода заключается в том, чтобы построить следующую последовательность: $y_{i+1}=f(y_{i})$, сходящейся к функции $y(x)$. Математической моделью в данном методе является единичная окружность с парой векторов, исходящих из центра.
	
	\begin{figure}[H]
    \centering
    \includegraphics[width=0.4\textwidth]{../image/cordic.png}
    \caption{Математическая модель CORDIC.}
	\end{figure}

	Вектор $V_{x}$ отклонён от горизонтальной оси на угол являющимся аргументов функции. Второй вектор $V_{0}$ будет производить вращение от начальной точки относительно начала координат. Координаты векторов имеют значения $sin$ и $cos$ угла, на который вектор отклоняется от горизонтальной оси. 

	Для вектора $V_{0}$: $cos\;0 = 1$, $sin\;0 = 0$. 

	Для вектора $V_{x}$: $cos\;\phi = x$, $sin\;\phi = y$.

	Необходимо найти координаты вектора $V_{x}$ $x$ и $y$ после поворота на угол $\phi$. Координаты вычисляются по тригонометрическим формулам:

	$x=x_{0}*cos\;\phi-y_{0}*sin\;\phi$,

	$y=x_{0}*sin\;\phi+y_{0}*cos\;\phi$

	Так как $tan\;\phi=\dfrac{sin\;\phi}{cos\;\phi}$, то можно выразить $sin\;\phi=tan\;\phi*cos\;\phi$ и выполнить преобразование формул. Тогда получим:

	$x=cos\;\phi(x_{0}-y_{0}*tan\;\phi)$,

	$y=cos\;\phi(y_{0}+x_{0}*tan\;\phi)$

	\begin{figure}[H]
    \centering
    \includegraphics[width=0.4\textwidth]{../image/cordic2.png}
    \caption{Поворот вектора.}
	\end{figure}

	Если задавать такой угол поворота, что $tan\;\phi = \pm 2^{-i}$, где $i$ --- целое число, то умножение $x_{0}$ и $y_{0}$ сведётся к простому сдвигу их значений вправо на $i$ разрядов, так как деление на 2 представляет из себя побитовый сдвиг числа право.

	Произвольный угол можно представить в виде суммы углов:

	$\phi_{i}=\pm atan2^{-i}$, где $i = 0, 1, 2,$ и т.д.

	Тогда операция поворота вектора будет состоять из последовательных простых поворотов. В каждой итерации проводятся следующие вычисления:

	$\sigma_{i} 	= sign(z_{i})$ --- направление поворота,

	$x_{i+1} = x_{i} - \sigma_{i}*y_{i}*2^{-i}$ --- значение координаты $x$,

	$y_{i+1} = y_{i} + \sigma_{i}*x_{i}*2^{-i}$ --- значение координаты $y$,

	$z_{i+1} = z_{i} - \sigma_{i}*atan(z^{-i})$ --- отклонение вектора.

	Данный алгоритм применим для генерации синуса и его применение целесообразно только при необходимости быстродействия и высокой точности системы.
	
\subsection*{Табличный метод}		
	В табличном методе генерации сигналов предполагается, что заранее вычисленные отсчёты хранятся в памяти. То есть никаких вычислений не требуется и генерация сводится к тому, что в порт цифро-аналогового преобразователя нужно вывести ячейку по заданному адресу. <<Достоинством этого метода является меньшее время, затрачиваемое на формирование отсчета и, как следствие, возможность генерации сигналов с более высокой частотой. Недостатком является необходимость иметь большой объем памяти данных.>>~\cite{leso}. 
	
	Частота сигнала будет зависеть от опорной частоты устройства.

	$f_{out}=\dfrac{f_{clk}}{n}$, где n --- количество отсчётов (длина таблицы).	
	
	\begin{figure}[H]
	\centering
    \includegraphics[width=1\textwidth]{../image/table_func.pdf}
    \caption{Функциональная схема табличного метода.}
	\end{figure}
	
	Управлять частотой устройства не всегда удобно. При желании уменьшить частоту сигнала придётся добавлять какую-то задержку в цикл, а что делать, если появилась необходимость увеличить частоту и код уже максимально оптимизирован. К примеру максимальная частота, которой удалось достигнуть 10 кГц и на большее наше устройство уже не способно. Так как увеличить частоту опроса таблицы уже невозможно, то нужно уменьшить её длину. То есть чтобы нам получить на выходе 20 кГц мы должны будем выводить каждый второй отсчёт таблицы, если 30 кГц, то каждый третий и т. д. Это хороший вариант, но тогда возникает проблема как дополнить программу, чтобы она пропускала нужное количество отсчётов.



\subsection*{Метод DDS}	

	К табличным методам относится также метод прямого цифрового синтеза или как его ещё называют метод DDS и он решает проблему, в которую упирается обычный табличный метод. <<Прямой цифровой синтез (от англ. DDS – Direct Digital Synthesizer) – метод, позволяющий получить аналоговый сигнал (обычно это синусоидальный сигнал, пилообразный, последовательность треугольных импульсов) за счет генерации временной последовательности цифровых отсчетов и их дальнейшего преобразования в аналоговую форму посредством ЦАП.>>~\cite{leso}. На рисунке 1.13 изображена функциональная схема DDS с аккумулятором фазы.
	
	Частота сигнала в этой архитектуре определяется следующей формулой:
	
	$f_{out}=\dfrac{D * f_{clk}}{2^{A}}$, где $f_{out}$ --- выходная частота, $f_{clk}$ --- частота устройства, $D$ --- код частоты, $A$ --- разрядность аккумулятора фазы.
	
	Благодаря разрядности аккумулятора фазы можно определять насколько точно будет регулироваться частота выходного сигнала.
	
	\begin{figure}[H]
    \centering
    \includegraphics[width=1\textwidth]{../image/dds_func.pdf}
    \caption{Функциональная схема DDS с аккумулятором фазы.}
	\end{figure}
	
	В аккумуляторе фазы и есть ключевое отличие метода DDS от простого табличного синтеза. Аккумулятор фазы представляет из себя регистр, в котором в каждом такте работы устройства происходит перезагрузка величины и прибавляется заданный код частоты. Приращение зависит как раз-таки от кода частоты и регулирует это значение. Таким образом, происходит вычисление какой отсчёт нужно отправить в порт цифро-аналогового преобразователя. Ещё одним отличием от табличного способа генерации является работа на фиксированной частоте. Алгоритм метода DDS можно описать блок-схемой на рисунке 1.14.
	
	\begin{figure}[H]
    \centering
    \includegraphics[width=0.35\textwidth]{../image/dds_block.pdf}
    \caption{Алгоритм метода DDS.}
	\end{figure}
	
	С помощью данного метода можно производить синтез не только стандартных форм сигналов, но и создавать произвольные формы. Метод DDS позволяет управлять цифровым способом амплитудой и фазой сигнала, а также лежит во основе многих приборов. <<Эти приборы находят широкое применение в различных устройствах: тестовом, измерительном, коммуникационном оборудовании. Интегральные DDS --- это компактные, потребляющие минимум электроэнергии, недорогие устройства и, кроме того, очень простые с точки зрения применения.>>~\cite{dds}. 
	
\section{Вывод из первой главы}
	Таким образом, можно сделать вывод о том, что среди генераторов сигналов наиболее  выделяются функциональные генераторы своей универсальностью и гибкостью. Они способны создавать различные функциональные зависимости, что позволяет генерировать сигналы разной формы, включая синусоидальные, треугольные и прямоугольные сигналы в широком спектре частот. Это делает их очень полезными для тестирования, исследования и отладки электронной аппаратуры. В следствие этого было принято решение разрабатывать функциональный генератор сигналов. В качестве метода генерации сигнала был выбран метод DDS за его простоту реализации и гибкость.
	
	
\chapter{ПРОЕКТИРОВАНИЕ}
\section{Моделирование DDS}
Для начала потребуется таблица отсчётов, чтобы её вычислить используем готовый инструмент~\cite{lut}.
	
	\begin{figure}[H]
    \centering
    \includegraphics[width=1\textwidth]{../image/lut_prog.png}
    \caption{Программа для вычисления значений сигнала.}
	\end{figure}
	
	У таблицы есть 4 параметра:
	\begin{enumerate}
		\item Разрядность ЦАП: 8 или 12 бит.
		\item Максимальное значение.
		\item Количество значений.
		\item Смещение от нуля.
	\end{enumerate}
	
	Использовать мы будем 12-битные значения в количестве 256 чисел. Максимальное значение амплитуды сигнала может быть 4095.%, но так как для улучшения генерации будет задействован встроенный в цифро-аналоговый преобразователь выходной буфер, то он будет срезать сигнал сверху и снизу на 0.2В, поэтому значения тоже следует срезать на эту же величину для корректной генерации.
	
%	В документе от ST про работу с цифро-аналоговым преобразователем есть формула для расчета выходного напряжения.
	
%	$DAC_{output} = V_{REF}*\dfrac{DOR}{DAC_{MaxDigitalValue} + 1}$, где DOR --- цифровое значение.
	
%	Нам нужно найти какое значение соответствует напряжению 0.2В. Выразим DOR и подставим имеющиеся значения.
	
%	$DOR = \dfrac{V_{REF}}{DOR}*DAC_{MaxDigitalValue} + 1 = \dfrac{3.3}{0.2}*(4095+1) = 248$
	
%	Укажем смещение от нуля 248, а максимальное значение 4095 меньше на 248, то есть 3847 и сгенеририуем таблицу отсчётов для синусоиды. 
	
	\begin{figure}[H]
    \centering
    \includegraphics[width=1\textwidth]{../image/lut.png}
    \caption{Вычисление таблицы сигнала.}
	\end{figure}
	
	Теперь у нас есть данные для генерации сигнала. Смоделируем алгоритм метода прямого цифрового синтеза по блок-схеме на рис. 1.15 на языке Си для дальнейшей реализации на микроконтроллере.

\begin{code}
\captionof{listing}{Метод DDS.}
\begin{minted}[mathescape,linenos,frame=lines,breaklines]{text}
int main() {
  uint16_t p_acc, p_step;
  uint8_t addr = 0; // адрес ячейки

  p_acc = 0;    // аккумулятор фазы
  p_step = 128; // код частоты

  while(1)
  {
    addr = p_acc >> 8; // выделение старшей части аккумулятора фазы
    p_acc += p_step;   // шаг
    printf("%d 0x%X\n", addr, sinus[addr]); // вывод отсчёта
  }

  return 0;
}
\end{minted}
\end{code}

	
	Код частоты задаёт выходную частоту генератора. При значении 256 вывод будет следующий:
	
\begin{figure}[H]
    \centering
    \includegraphics[width=0.6\textwidth]{../image/dds256.png}
    \caption{Формирование отсчётов при коде частоты 256.}
\end{figure}
	
	Увеличим код частоты в два раза и получим следующее:

\begin{figure}[H]
    \centering
    \includegraphics[width=0.6\textwidth]{../image/dds512.png}
    \caption{Формирование отсчётов при коде частоты 512.}
\end{figure}

	Как можно заметить отсчёты стали формироваться через один, соответственно частота вырастит в два раза. Теперь уменьшим частоту в два раза выставив код частоты 128.

\begin{figure}[H]
    \centering
    \includegraphics[width=0.6\textwidth]{../image/dds128.png}
    \caption{Формирование отсчётов при коде частоты 128.}
\end{figure}

	Программа стала выводить каждый отсчёт по два раза тем самым, понизив частоту.
	
	В данном виде модуляции код частоты просто абстрактное число, которое добавляется к аккумулятору фазы и узнать реальную частоту проблематично. Результат синтеза будет проверен опытным путём на микроконтроллере.
	
\section{Обзор микроконтроллеров}
	Так как генератор сигналов будет реализовываться на микроконтроллере следует провести обзор и осуществить выбор. Рассмотрим два популярных семейства микроконтроллеров AVR и STM32.
\subsection{AVR}
	Микроконтроллеры AVR --- это 8-разрядные микроконтроллеры с архитектурой RISC. Данное семейство представляет собой хорошую основу для создания высокопроизводительных и экономичных встраиваемых систем~\cite{avr}. Подразделяется семейство на две группы: Tiny и Mega.
	
	Микроконтроллеры Tiny имеют небольшую память для программ и их периферия ограничена. Большинство микроконтроллеров данной серии выпускаются в 8-выводных корпусах и предназначены для систем с ограниченным бюджетом. Областью их применения являются различные датчики и бытовая техника~\cite{avr}.
	
	Группа Mega наоборот имеет большую память и развитую периферию. Соответственно область применения гораздо шире и предназначены они для более сложных систем. В таблице 2.1 приведены серии микроконтроллеров и коротко описан их приоритет.

\begin{table}[H]
\caption{Микроконтроллеры AVR.}
\begin{tabular}{|p{3.25 cm}|p{8 cm}|p{4 cm}|}
\hline
Группа & Приоритет & Название серий \\ \hline
Tiny & Энергоэффективность, компактность, низкая стоимость & tiny1, tiny2, tiny4, tiny8 \\ \hline
Mega & Производительность, гибкость & mega4, mega8, mega16, mega32, mega64, mega128, mega256 \\ \hline
\end{tabular}
\end{table}

\subsection{STM32}
	Микроконтроллеры STM32 --- это 32-разрядные микроконтроллеры, имеющие процессорное ядро с архитектурой ARM Cortex-M. В настоящее время существует множество микроконтроллеров STM32. Они делятся на семейства в зависимости от версии архитектуры (табл. 2.2).

\begin{table}[H]
\caption{Семейства STM32.}
\begin{tabular}{|p{4 cm}|p{4 cm}|}
\hline
Серия & Ядро \\ \hline
F0  & Cortex-М0 \\ \hline
G0, L0  & Cortex-М0+ \\ \hline
F1, F2  & Cortex-М3 \\ \hline
F3, F4, L4, G4  & Cortex-М4 \\ \hline
F7, H7  & Cortex-М7 \\ \hline
\end{tabular}
\end{table}

	Ядро Cortex-M обеспечивает программную совместимость во всех семействах. Кроме этого, для микроконтроллеров выпущенных в одинаковых корпусах присутствует и аппаратная совместимость, так как на выводах сохраняются одни и те же функции~\cite{stm}. Будем рассматривать серии микроконтроллеров схожие по функциональным возможностям с Tiny и Mega для дальнейшего сравнения. В таблице 2.3 указаны серии STM32 по группам.
	
\begin{table}[H]
\caption{Микроконтроллеры STM32.}
\begin{tabular}{|p{3.25 cm}|p{8 cm}|p{4 cm}|}
\hline
Группа & Приоритет & Название серий \\ \hline
Широкого применения & Баланс между производительностью и энергоэффективностью & F0, G0, F1, F3,
G4 \\ \hline
Сверхнизкого энергопотребления & Энергоэффективность, компактность, низкая стоимость & L0, L4 \\ \hline
\end{tabular}
\end{table}

\section{Сравнение семейств AVR и STM32}
	
	Для осуществления выбора проведём сравнение микроконтроллеров, взяв параметры наиболее используемых серий из каждой группы (табл. 2.4).
	
\begin{table}[H]
\caption{Параметры микроконтроллеров.}
\begin{tabular}{|p{2.5 cm}|p{3 cm}|p{3 cm}|p{3 cm}|p{3 cm}|}
\hline
        Параметр & ATtiny1 & ATmega32 & STM32L010 & STM32F103 \\ \hline
        Частота & 20 МГц & 20 МГц & 32 МГц & 72 МГц \\ \hline
        FLASH & 1 Кбайт & 32 Кбайт & 16 Кбайт & 64 Кбайт \\ \hline
        RAM & 64 байт & 2 Кбайт & 2 Кбайт & 20 Кбайт \\ \hline
        SPI & - & + & + & + \\ \hline
        I2C & - & +	 & + & + \\ \hline
        Питание & 1,8 --- 5,5 В & 1,8 --- 5,5 В & 1,8 --- 3,6 В & 1,8 --- 3,6 В \\ \hline
\end{tabular}
\end{table}
	
	Исходя из таблицы можно сделать вывод, что микроконтроллеры AVR применимы в малом спектре задач где скорость не так важна. В нашем же случае скорость работы микроконтроллера может сильно влиять на генерацию сигнала, а также требуется объём памяти для хранения отсчётов сигналов. В микроконтроллерах STM32 с частотой и объёмом памяти проблем нет и они имеют широкое применение. Серию же выберем F103 за её характеристики. В связи с этим в устройстве будет применён микроконтроллер STM32F103RCT6.

\section{Среды разработки для STM32}

	Среда разработки является не маловажным инструментом для создания программной части устройства. В связи с выбором микроконтроллера STM32 рассмотрим популярные бесплатные среды для создания программы на этом семействе микроконтроллеров.

\subsection{STM32CubeIDE}
	STM32CubeIDE --- это продвинутая платформа разработки на C/C++ с функциями настройки периферийных устройств, генерации кода, компиляции кода и отладки для микроконтроллеров и микропроцессоров STM32~\cite{cube}. Среда разработки основана на платформе Eclipse и GCC toolchain для разработки и GDB для отладки. Она позволяет интегрировать сотни существующих плагинов, которые дополняют возможности Eclipse IDE. Имеет расширенные функции отладки, включая: просмотр ядра ЦП, регистров периферийных устройств и памяти, анализ системы просмотра переменных в режиме реального времени. Поддерживается на операционных системах: Linux, macOS, Windows.

	\begin{figure}[H]
    \centering
    \includegraphics[width=0.85\textwidth]{../image/cube.jpg}
    \caption{Интерфейс STM32CubeIDE.}
	\end{figure}
	
	После выбора микроконтроллера STM32 создается проект и генерируется код инициализации. В любой момент разработки пользователь может вернуться к инициализации и настройке периферийных устройств и повторно создать код инициализации без какого-либо влияния на пользовательский код. Для разработки используется библиотека HAL. 
	
	Драйверы HAL включают в себя полный набор готовых к использованию функций, которые упрощают реализацию пользовательских приложений. Например, коммуникационные периферийные устройства содержат функции для инициализации и настройки периферийного устройства, управления передачей данных, обработки прерываний или DMA~\cite{hal}.
	
	Достоинства:
	\begin{itemize}
		\item Поддержка различных ОС.
		\item Расширенные возможности отладки.
		\item Большое сообщество.
		\item Автогенерация кода.
	\end{itemize}
	
	Недостатки:
	\begin{itemize}
		\item Требовательность к ресурсам ПК.
		\item Сложность настройки.
	\end{itemize}

\subsection{PlatformIO}

	PlatformIO --- удобная и расширяемая интегрированная среда разработки с набором профессиональных инструментов разработки, предоставляющая современные и мощные функции для ускорения и упрощения процесса разработки встраиваемых продуктов~\cite{plio}.
	
	\begin{figure}[H]
    \centering
    \includegraphics[width=0.85\textwidth]{../image/plio.jpg}
    \caption{Интерфейс PlatformIO.}
	\end{figure}
	
	Данная среда разработки является расширением для текстового редактора Visual Studio Code. VS Code --- это легкий, но мощный редактор кода, имеющий богатую экосистему расширений~\cite{docsplio}. Доступен для Windows, macOS и Linux. Работа в паре с VS Code позволяет удобно форматировать код и пользоваться расширениями для языков программирования.
	
	PlatformIO позволяет работать со многими микроконтроллерами и поддерживает множество фреймворков для них, а также библиотек. Ввиду такой широкой поддержки, для STM32 можно разрабатывать с удобной для себя библиотекой. Это может быть к примеру тот же HAL, что и в STM32CubeIDE или libopencm3. Проект libopencm3 (ранее известный как libopenstm32) направлен на создание бесплатной библиотеки микропрограмм с открытым исходным кодом (LGPL версии 3 или более поздней) для различных микроконтроллеров ARM Cortex-M3, включая ST STM32~\cite{lcm3}.


	Достоинства:
	\begin{itemize}
		\item Поддержка различных ОС.
		\item Быстрая компиляция.
		\item Поддержка GitHub.
		\item Возможность работать с разными фрэймворками и платформами.
	\end{itemize}
	
	Недостатки:
	\begin{itemize}
		\item Высокий порог вхождения.
		\item Сложность установки.
	\end{itemize}

	Попользовавшись обеими средами разработки и разными библиотеками, а также основываясь на достоинствах и недостатках была выбрана среда разработки PlatformIO в связке с библиотекой libopencm3.


\section{Алгоритм работы}

	Программа должна выполнять три действия:
	
	\begin{enumerate}
		\item Вывод отсчёта в ЦАП.
		\item Обработка кнопок.
		\item Вывод информации на дисплей.
	\end{enumerate}
	
	Для цифро-аналогового преобразователя и кнопок выделим два таймера общего назначения, а работа с дисплеем будет идти в главном цикле программы. Применив такой подход, удастся добиться асинхронного выполнения программы. 
	
	Таким образом, для подпрограммы генерации сигнала будет следующая блок-схема.
	
	\begin{figure}[H]
    \centering
    \includegraphics[width=0.3\textwidth]{../image/dac.pdf}
    \caption{Блок-схема функции ЦАП.}
	\end{figure}
	
	По созданной блок-схеме код 
	
\begin{code}
\captionof{listing}{Генерация сигнала.}
\begin{minted}[mathescape,linenos,frame=lines,breaklines]{text}
void tim2_isr(void) // обработчик прерывания таймера2 (ЦАП)
{
    dac_load_data_buffer_single(signal[p_acc >> 8], RIGHT12, CHANNEL_2); // загрузка буфера в цап
    p_acc += p_step;             // шаг
    TIM_SR(TIM2) &= ~TIM_SR_UIF; // очистка флага прерывания
}
\end{minted}
\end{code}
	
	Обработка кнопок представлена следующей блок-схемой.	
	
	\begin{figure}[H]
    \centering
    \includegraphics[width=0.3\textwidth]{../image/buttons.pdf}
    \caption{Блок-схема функции кнопок.}
	\end{figure}
	
	Подпрограмма обработки кнопок находится в обработчике прерывания таймера номер 3. Таймер настроен на период 250 миллисекунд. Благодаря такой организации, решается проблема дребезга кнопок. Не приходится делать программную или аппаратную задержку для ожидания установки состояния кнопки.

\begin{code}
\captionof{listing}{Обработка кнопок.}
\begin{minted}[mathescape,linenos,frame=lines,breaklines]{text}
void tim3_isr(void) // обработчик прерывания таймера3 (обработка кнопок)
{
    minus_freq();
    plus_freq();
    minus_signal(); // функции кнопок
    plus_signal();
    step_select();
    TIM_SR(TIM3) &= ~TIM_SR_UIF; // очистка флага прерывания
}
\end{minted}
\end{code}


\section{Вывод из второй главы}

\chapter{Глава 3}
%\section{Сборка макета}

%\section{Кодовые фрагменты}

%\section{Тестирование на осциллографе}

\section{Вывод из третьей главы}

\chapter*{ЗАКЛЮЧЕНИЕ}
	В результате выполнения данной выпускной квалификационной работы была достигнута поставленная цель --- разработан программный генератор сигналов на микроконтроллере STM32F103RCT6, позволяющий генерировать сигналы разной формы, со следующими характеристиками:

	\begin{itemize}
		\item Формы сигналов: синус, треугольник, прямоугольник, пилообразная, обратная пилообразная.
		\item Частота сигнала: 125 --- 50000 Гц.
		\item Амплитуда: 3 В.
		\item Шаг по частоте: 125, 250, 500, 1000 Гц.
	\end{itemize}

	Помимо микроконтроллера генератор состоит из дисплея с разрешением 128 на 64 пикселя, работающего по интерфейсу I2C, и пяти кнопок управления.

	Для достижения поставленной цели были выполнены все задачи, а именно:
	\begin{enumerate}
		\item Выбран метод генерации сигналов;
		\item Выбран микроконтроллер;
		\item Выбрана среда разработки;
		\item Разработана программа;
		\item Спроектировано устройство;
		\item Протестирован генератор.
	\end{enumerate}

	Реализованный генератор сигналов отличается простотой, так как использует встроенный цифро-аналоговый преобразователь микроконтроллера и тем самым компактен, а также доступные элементы периферии ввиду этого также его плюсом является невысокая стоимость. 
\addcontentsline{toc}{chapter}{ЗАКЛЮЧЕНИЕ}




\newpage
\addcontentsline{toc}{chapter}{СПИСОК ИСПОЛЬЗОВАННОЙ ЛИТЕРАТУРЫ}
\printbibliography[title={СПИСОК ИСПОЛЬЗОВАННОЙ ЛИТЕРАТУРЫ}]

\appendix
\newpage
\addcontentsline{toc}{chapter}{Приложение}
\begin{flushright}
\uppercase{Приложение}\label{appendix}
\end{flushright}
\begin{code}
\captionof*{listing}{Программа генератора сигналов.}
\begin{minted}[mathescape,linenos,frame=lines,breaklines]{text}
#include <stdio.h>
#include <wchar.h>
#include <libopencm3/stm32/rcc.h>
#include <libopencm3/stm32/flash.h>
#include <libopencm3/stm32/gpio.h>
#include <libopencm3/stm32/timer.h>
#include <libopencm3/cm3/nvic.h>
#include <libopencm3/stm32/dac.h>
#include <libopencm3/stm32/i2c.h>
#include <ssd1306_i2c.h>

static void gpio_setup(void);   // установить входы/выходы
static void dac_setup(void);    // настройка цап
static void i2c_setup(void);    // настройка и2ц
static void timers_setup(void); // настройка таймеров
static void nvic_setup(void);   // настройка прерываний
void minus_freq(void);          //
void plus_freq(void);           //
void minus_signal(void);        // функции для кнопок
void plus_signal(void);         //
void step_select(void);         //

uint16_t p_acc = 0;         // аккумулятор фазы
int p_step = 0;             // код частоты 192 - 1khz
uint16_t step = 0;          // размер шага
uint16_t signal[256] = {0}; // буфер для цапа
int8_t num_sig = 0;         // номер сигнала
int8_t num_step = 0;        // номер шага

/*  отсчеты сигналов    */
uint16_t sinus[256] = {2048, 2092, 2136, 2180, 2224, 2268, 2312, 2355, 2399, 2442,
                       2485, 2527, 2570, 2612, 2654, 2695, 2736, 2777, 2817, 2857, 2896, 2934, 2973, 3010, 3047,
                       3084, 3119, 3155, 3189, 3223, 3256, 3288, 3320, 3351, 3381, 3410, 3439, 3466, 3493, 3519,
                       3544, 3568, 3591, 3613, 3635, 3655, 3674, 3693, 3710, 3726, 3742, 3756, 3770, 3782, 3793,
                       3803, 3812, 3821, 3828, 3833, 3838, 3842, 3845, 3846, 3847, 3846, 3845, 3842, 3838, 3833,
                       3828, 3821, 3812, 3803, 3793, 3782, 3770, 3756, 3742, 3726, 3710, 3693, 3674, 3655, 3635,
                       3613, 3591, 3568, 3544, 3519, 3493, 3466, 3439, 3410, 3381, 3351, 3320, 3288, 3256, 3223,
                       3189, 3155, 3119, 3084, 3047, 3010, 2973, 2934, 2896, 2857, 2817, 2777, 2736, 2695, 2654,
                       2612, 2570, 2527, 2485, 2442, 2399, 2355, 2312, 2268, 2224, 2180, 2136, 2092, 2048, 2003,
                       1959, 1915, 1871, 1827, 1783, 1740, 1696, 1653, 1610, 1568, 1525, 1483, 1441, 1400, 1359,
                       1318, 1278, 1238, 1199, 1161, 1122, 1085, 1048, 1011, 976, 940, 906, 872, 839, 807, 775,
                       744, 714, 685, 656, 629, 602, 576, 551, 527, 504, 482, 460, 440, 421, 402, 385, 369, 353,
                       339, 325, 313, 302, 292, 283, 274, 267, 262, 257, 253, 250, 249, 248, 249, 250, 253, 257,
                       262, 267, 274, 283, 292, 302, 313, 325, 339, 353, 369, 385, 402, 421, 440, 460, 482, 504,
                       527, 551, 576, 602, 629, 656, 685, 714, 744, 775, 807, 839, 872, 906, 940, 976, 1011, 1048,
                       1085, 1122, 1161, 1199, 1238, 1278, 1318, 1359, 1400, 1441, 1483, 1525, 1568, 1610, 1653,
                       1696, 1740, 1783, 1827, 1871, 1915, 1959, 2003};
/*                      */
uint16_t square[256] = {3847, 3847, 3847, 3847, 3847, 3847, 3847, 3847, 3847, 3847,
                        3847, 3847, 3847, 3847, 3847, 3847, 3847, 3847, 3847, 3847, 3847, 3847, 3847, 3847,
                        3847, 3847, 3847, 3847, 3847, 3847, 3847, 3847, 3847, 3847, 3847, 3847, 3847, 3847,
                        3847, 3847, 3847, 3847, 3847, 3847, 3847, 3847, 3847, 3847, 3847, 3847, 3847, 3847,
                        3847, 3847, 3847, 3847, 3847, 3847, 3847, 3847, 3847, 3847, 3847, 3847, 3847, 3847,
                        3847, 3847, 3847, 3847, 3847, 3847, 3847, 3847, 3847, 3847, 3847, 3847, 3847, 3847,
                        3847, 3847, 3847, 3847, 3847, 3847, 3847, 3847, 3847, 3847, 3847, 3847, 3847, 3847,
                        3847, 3847, 3847, 3847, 3847, 3847, 3847, 3847, 3847, 3847, 3847, 3847, 3847, 3847,
                        3847, 3847, 3847, 3847, 3847, 3847, 3847, 3847, 3847, 3847, 3847, 3847, 3847, 3847,
                        3847, 3847, 3847, 3847, 3847, 3847, 248, 248, 248, 248, 248, 248, 248, 248, 248, 248,
                        248, 248, 248, 248, 248, 248, 248, 248, 248, 248, 248, 248, 248, 248, 248, 248, 248,
                        248, 248, 248, 248, 248, 248, 248, 248, 248, 248, 248, 248, 248, 248, 248, 248, 248,
                        248, 248, 248, 248, 248, 248, 248, 248, 248, 248, 248, 248, 248, 248, 248, 248, 248,
                        248, 248, 248, 248, 248, 248, 248, 248, 248, 248, 248, 248, 248, 248, 248, 248, 248,
                        248, 248, 248, 248, 248, 248, 248, 248, 248, 248, 248, 248, 248, 248, 248, 248, 248,
                        248, 248, 248, 248, 248, 248, 248, 248, 248, 248, 248, 248, 248, 248, 248, 248, 248,
                        248, 248, 248, 248, 248, 248, 248, 248, 248, 248, 248, 248, 248, 248, 248, 248};
/*                      */
uint16_t triangle[256] = {248, 276, 304, 332, 360, 389, 417, 445, 473, 501, 529, 557, 585, 614, 642, 670, 698,
                          726, 754, 782, 810, 838, 867, 895, 923, 951, 979, 1007, 1035, 1063, 1092, 1120, 1148, 1176, 1204, 1232, 1260,
                          1288, 1316, 1345, 1373, 1401, 1429, 1457, 1485, 1513, 1541, 1570, 1598, 1626, 1654, 1682, 1710, 1738, 1766, 1794,
                          1823, 1851, 1879, 1907, 1935, 1963, 1991, 2019, 2048, 2076, 2104, 2132, 2160, 2188, 2216, 2244, 2272, 2301, 2329,
                          2357, 2385, 2413, 2441, 2469, 2497, 2525, 2554, 2582, 2610, 2638, 2666, 2694, 2722, 2750, 2779, 2807, 2835, 2863,
                          2891, 2919, 2947, 2975, 3003, 3032, 3060, 3088, 3116, 3144, 3172, 3200, 3228, 3257, 3285, 3313, 3341, 3369, 3397,
                          3425, 3453, 3481, 3510, 3538, 3566, 3594, 3622, 3650, 3678, 3706, 3735, 3763, 3791, 3819, 3847, 3819, 3791, 3763,
                          3735, 3706, 3678, 3650, 3622, 3594, 3566, 3538, 3510, 3481, 3453, 3425, 3397, 3369, 3341, 3313, 3285, 3257, 3228,
                          3200, 3172, 3144, 3116, 3088, 3060, 3032, 3003, 2975, 2947, 2919, 2891, 2863, 2835, 2807, 2779, 2750, 2722, 2694,
                          2666, 2638, 2610, 2582, 2554, 2525, 2497, 2469, 2441, 2413, 2385, 2357, 2329, 2301, 2272, 2244, 2216, 2188, 2160,
                          2132, 2104, 2076, 2048, 2019, 1991, 1963, 1935, 1907, 1879, 1851, 1823, 1794, 1766, 1738, 1710, 1682, 1654, 1626,
                          1598, 1570, 1541, 1513, 1485, 1457, 1429, 1401, 1373, 1345, 1316, 1288, 1260, 1232, 1204, 1176, 1148, 1120, 1092,
                          1063, 1035, 1007, 979, 951, 923, 895, 867, 838, 810, 782, 754, 726, 698, 670, 642, 614, 585, 557, 529, 501, 473,
                          445, 417, 389, 360, 332, 304, 276};
/*                      */
uint16_t l_saw[256] = {248, 262, 276, 290, 304, 319, 333, 347, 361, 375, 389, 403, 417, 431, 446, 460, 474,
                       488, 502, 516, 530, 544, 559, 573, 587, 601, 615, 629, 643, 657, 671, 686, 700, 714, 728, 742, 756, 770, 784,
                       798, 813, 827, 841, 855, 869, 883, 897, 911, 925, 940, 954, 968, 982, 996, 1010, 1024, 1038, 1052, 1067, 1081,
                       1095, 1109, 1123, 1137, 1151, 1165, 1180, 1194, 1208, 1222, 1236, 1250, 1264, 1278, 1292, 1307, 1321, 1335, 1349,
                       1363, 1377, 1391, 1405, 1419, 1434, 1448, 1462, 1476, 1490, 1504, 1518, 1532, 1546, 1561, 1575, 1589, 1603, 1617,
                       1631, 1645, 1659, 1673, 1688, 1702, 1716, 1730, 1744, 1758, 1772, 1786, 1801, 1815, 1829, 1843, 1857, 1871, 1885,
                       1899, 1913, 1928, 1942, 1956, 1970, 1984, 1998, 2012, 2026, 2040, 2055, 2069, 2083, 2097, 2111, 2125, 2139, 2153,
                       2167, 2182, 2196, 2210, 2224, 2238, 2252, 2266, 2280, 2294, 2309, 2323, 2337, 2351, 2365, 2379, 2393, 2407, 2422,
                       2436, 2450, 2464, 2478, 2492, 2506, 2520, 2534, 2549, 2563, 2577, 2591, 2605, 2619, 2633, 2647, 2661, 2676, 2690,
                       2704, 2718, 2732, 2746, 2760, 2774, 2788, 2803, 2817, 2831, 2845, 2859, 2873, 2887, 2901, 2915, 2930, 2944, 2958,
                       2972, 2986, 3000, 3014, 3028, 3043, 3057, 3071, 3085, 3099, 3113, 3127, 3141, 3155, 3170, 3184, 3198, 3212, 3226,
                       3240, 3254, 3268, 3282, 3297, 3311, 3325, 3339, 3353, 3367, 3381, 3395, 3409, 3424, 3438, 3452, 3466, 3480, 3494,
                       3508, 3522, 3536, 3551, 3565, 3579, 3593, 3607, 3621, 3635, 3649, 3664, 3678, 3692, 3706, 3720, 3734, 3748, 3762,
                       3776, 3791, 3805, 3819, 3833, 3847};
/*                      */
uint16_t r_saw[256] = {3847, 3833, 3819, 3805, 3791, 3776, 3762, 3748, 3734, 3720, 3706, 3692, 3678, 3664,
                       3649, 3635, 3621, 3607, 3593, 3579, 3565, 3551, 3536, 3522, 3508, 3494, 3480, 3466, 3452, 3438, 3424, 3409,
                       3395, 3381, 3367, 3353, 3339, 3325, 3311, 3297, 3282, 3268, 3254, 3240, 3226, 3212, 3198, 3184, 3170, 3155,
                       3141, 3127, 3113, 3099, 3085, 3071, 3057, 3043, 3028, 3014, 3000, 2986, 2972, 2958, 2944, 2930, 2915, 2901,
                       2887, 2873, 2859, 2845, 2831, 2817, 2803, 2788, 2774, 2760, 2746, 2732, 2718, 2704, 2690, 2676, 2661, 2647,
                       2633, 2619, 2605, 2591, 2577, 2563, 2549, 2534, 2520, 2506, 2492, 2478, 2464, 2450, 2436, 2422, 2407, 2393,
                       2379, 2365, 2351, 2337, 2323, 2309, 2294, 2280, 2266, 2252, 2238, 2224, 2210, 2196, 2182, 2167, 2153, 2139,
                       2125, 2111, 2097, 2083, 2069, 2055, 2040, 2026, 2012, 1998, 1984, 1970, 1956, 1942, 1928, 1913, 1899, 1885,
                       1871, 1857, 1843, 1829, 1815, 1801, 1786, 1772, 1758, 1744, 1730, 1716, 1702, 1688, 1673, 1659, 1645, 1631,
                       1617, 1603, 1589, 1575, 1561, 1546, 1532, 1518, 1504, 1490, 1476, 1462, 1448, 1434, 1419, 1405, 1391, 1377,
                       1363, 1349, 1335, 1321, 1307, 1292, 1278, 1264, 1250, 1236, 1222, 1208, 1194, 1180, 1165, 1151, 1137, 1123,
                       1109, 1095, 1081, 1067, 1052, 1038, 1024, 1010, 996, 982, 968, 954, 940, 925, 911, 897, 883, 869, 855, 841,
                       827, 813, 798, 784, 770, 756, 742, 728, 714, 700, 686, 671, 657, 643, 629, 615, 601, 587, 573, 559, 544, 530,
                       516, 502, 488, 474, 460, 446, 431, 417, 403, 389, 375, 361, 347, 333, 319, 304, 290, 276, 262, 248};
/*                      */
void tim2_isr(void) // обработчик прерывания таймера2 (ЦАП)
{
    dac_load_data_buffer_single(signal[p_acc >> 8], RIGHT12, CHANNEL_2); // загрузка буфера в цап
    p_acc += p_step;                                                     // шаг
    TIM_SR(TIM2) &= ~TIM_SR_UIF;                                         // очистка флага прерывания
}

void tim3_isr(void) // обработчик прерывания таймера3 (обработка кнопок)
{
    minus_freq();
    plus_freq();
    minus_signal(); // функции кнопок
    plus_signal();
    step_select();
    TIM_SR(TIM3) &= ~TIM_SR_UIF; // очистка флага прерывания
}

int main(void)
{
    rcc_clock_setup_in_hse_8mhz_out_72mhz(); // установка тактирования
    gpio_setup();
    nvic_setup();
    dac_setup();
    timers_setup();
    i2c_setup();
    ssd1306_init(I2C2, DEFAULT_7bit_OLED_SLAVE_ADDRESS, 128, 64); // инициализация дисплея

    int f = 0;       // переменная частоты
    wchar_t freq[8]; // буфер для wchar_t строки
    while (1)
    {
        f = p_step / 24 * 125;
        swprintf(freq, sizeof(freq) / sizeof(wchar_t), L"%d", f); // Использование swprintf для преобразования int в wchar_t*
        /*  вывод информации на дисплей  */
        ssd1306_clear();
        ssd1306_drawWCharStr(0, 0, white, nowrap, L"Форма сигнала:");
        switch (num_sig)
        {
        case 1:
            ssd1306_drawWCharStr(0, 8, white, nowrap, L"Синус");
            break;
        case 2:
            ssd1306_drawWCharStr(0, 8, white, nowrap, L"Меандр");
            break;
        case 3:
            ssd1306_drawWCharStr(0, 8, white, nowrap, L"Треугольник");
            break;
        case 4:
            ssd1306_drawWCharStr(0, 8, white, nowrap, L"Пила Левая");
            break;
        case 5:
            ssd1306_drawWCharStr(0, 8, white, nowrap, L"Пила Правая");
            break;
        }
        ssd1306_drawWCharStr(0, 16, white, nowrap, L"Частота(Гц)");
        ssd1306_drawWCharStr(64, 16, white, nowrap, freq);
        ssd1306_drawWCharStr(0, 32, white, nowrap, L"Шаг(Гц)");
        switch (num_step)
        {
        case 1:
            ssd1306_drawWCharStr(64, 32, white, nowrap, L"125");
            break;
        case 2:
            ssd1306_drawWCharStr(64, 32, white, nowrap, L"250");
            break;
        case 3:
            ssd1306_drawWCharStr(64, 32, white, nowrap, L"500");
            break;
        case 4:
            ssd1306_drawWCharStr(64, 32, white, nowrap, L"1000");
            break;
        }
        ssd1306_refresh();
    }

    return 0;
}

static void gpio_setup(void)
{
    // rcc_periph_clock_enable(RCC_GPIOD); // тактирование портов
    rcc_periph_clock_enable(RCC_GPIOB);
    gpio_set_mode(GPIOB, GPIO_MODE_INPUT, GPIO_CNF_INPUT_PULL_UPDOWN, GPIO9 | GPIO5 | GPIO6 | GPIO7 | GPIO8); // входы для кнопок, подтянуты к земле
    // gpio_set_mode(GPIOD, GPIO_MODE_OUTPUT_50_MHZ, GPIO_CNF_OUTPUT_PUSHPULL, GPIO2);
}

static void dac_setup(void)
{
    rcc_periph_clock_enable(RCC_GPIOA);
    gpio_set_mode(GPIOA, GPIO_MODE_OUTPUT_2_MHZ, GPIO_CNF_OUTPUT_ALTFN_PUSHPULL, GPIO5);
    rcc_periph_clock_enable(RCC_DAC); // тактирование цапа и настройка вывода
    dac_enable(CHANNEL_2);            // включить цап
}

static void i2c_setup(void){
    // Включение тактирования периферийного оборудования для I2C2
    rcc_periph_clock_enable(RCC_I2C2);
    
    /*
     * Настройка альтернативных функций для пинов SCL и SDA интерфейса I2C2.
     * Это необходимо для подключения I2C устройств к микроконтроллеру через эти
 пины.
     */
    gpio_set_mode(GPIOB, GPIO_MODE_OUTPUT_50_MHZ,
                  GPIO_CNF_OUTPUT_ALTFN_OPENDRAIN,
                  GPIO_I2C2_SCL | GPIO_I2C2_SDA);
    
    // Отключение I2C перед изменением конфигурации
    i2c_peripheral_disable(I2C2);
    
    // Сброс состояния периферийного устройства I2C2
    i2c_reset(I2C2);
    
    // Установка стандартного режима работы I2C 
    i2c_set_standard_mode(I2C2);
    
    // Установка частоты периферии
    i2c_set_clock_frequency(I2C2, I2C_CR2_FREQ_36MHZ);
    
    // Настройка тактовой частоты шины; 
    i2c_set_ccr(I2C2, 0xB4);
    
    // Установка времени подъема сигнала SDA после завершения операции чтения/записи
    i2c_set_trise(I2C2, 0x25);
    
    // Включение подтверждения при получении данных от устройства
    i2c_enable_ack(I2C2);
    
    // Включение периферийного устройства I2C2
    i2c_peripheral_enable(I2C2);
}


static void timers_setup(void)
{
    rcc_periph_clock_enable(RCC_TIM2);
    rcc_periph_clock_enable(RCC_TIM3);

    /* Стартовое значение таймера */
    TIM_CNT(TIM2) = 0;
    TIM_CNT(TIM3) = 0;

    /* Предделитель 36MHz/36000 => 1000 отсчетов в секунду (счет начинается с 0, поэтому в предделителе и периоде нужно отнимать единичку) */
    TIM_PSC(TIM2) = 17;
    TIM_PSC(TIM3) = 35999;

    /* Период таймера */
    TIM_ARR(TIM2) = 9;
    TIM_ARR(TIM3) = 249;

    /* Включить прерывания */
    TIM_DIER(TIM2) |= TIM_DIER_UIE;
    TIM_DIER(TIM3) |= TIM_DIER_UIE;

    /* Запустить таймер */
    TIM_CR1(TIM2) |= TIM_CR1_CEN;
    TIM_CR1(TIM3) |= TIM_CR1_CEN;
}

static void nvic_setup(void)
{
    /* Активировать прерывания и установить приоритеты */
    nvic_enable_irq(NVIC_TIM2_IRQ);
    nvic_set_priority(NVIC_TIM2_IRQ, 2);

    nvic_enable_irq(NVIC_TIM3_IRQ);
    nvic_set_priority(NVIC_TIM3_IRQ, 1);
}

void minus_freq(void)
{
    bool cur_val = 0;
    bool prev_val = 0;
    cur_val = gpio_get(GPIOB, GPIO5);
    if (cur_val == 1 && prev_val == 0)
    {
        p_step -= step;
    }
    if (p_step < 0) // ограничение 0
    {
        p_step = 0;
    }
    prev_val = cur_val;
}

void plus_freq(void)
{
    bool cur_val = 0;
    bool prev_val = 0;
    cur_val = gpio_get(GPIOB, GPIO6);
    if (cur_val == 1 && prev_val == 0)
    {
        p_step += step;
    }
    if (p_step > 9600) // ограничение 50 кГц
    {
        p_step = 9600;
    }
    prev_val = cur_val;
}

void minus_signal(void)
{
    bool cur_val = 0;
    bool prev_val = 0;
    cur_val = gpio_get(GPIOB, GPIO7);
    if (cur_val == 1 && prev_val == 0)
    {
        dac_disable(CHANNEL_2);
        num_sig -= 1;
        if (num_sig < 1)
            num_sig = 1;
        switch (num_sig)
        {
        case 1:
            for (int i = 0; i < 256; i++)
                signal[i] = sinus[i];
            break;
        case 2:
            for (int i = 0; i < 256; i++)
                signal[i] = square[i];
            break;
        case 3:
            for (int i = 0; i < 256; i++)
                signal[i] = triangle[i];
            break;
        case 4:
            for (int i = 0; i < 256; i++)
                signal[i] = l_saw[i];
            break;
        case 5:
            for (int i = 0; i < 256; i++)
                signal[i] = r_saw[i];
            break;
        }
        dac_enable(CHANNEL_2);
    }
    prev_val = cur_val;
}

void plus_signal(void)
{
    bool cur_val = 0;
    bool prev_val = 0;
    cur_val = gpio_get(GPIOB, GPIO8);
    if (cur_val == 1 && prev_val == 0)
    {
        dac_disable(CHANNEL_2);
        num_sig += 1;
        switch (num_sig)
        {
        case 1:
            for (int i = 0; i < 256; i++)
                signal[i] = sinus[i];
            break;
        case 2:
            for (int i = 0; i < 256; i++)
                signal[i] = square[i];
            break;
        case 3:
            for (int i = 0; i < 256; i++)
                signal[i] = triangle[i];
            break;
        case 4:
            for (int i = 0; i < 256; i++)
                signal[i] = l_saw[i];
            break;
        case 5:
            for (int i = 0; i < 256; i++)
                signal[i] = r_saw[i];
            break;
        }
        if (num_sig > 5)
            num_sig = 5;
        dac_enable(CHANNEL_2);
    }
    prev_val = cur_val;
}

void step_select(void)
{
    bool cur_val = 0;
    bool prev_val = 0;
    cur_val = gpio_get(GPIOB, GPIO9);
    if (cur_val == 1 && prev_val == 0)
    {
        num_step += 1;
        switch (num_step)
        {
        case 1:
            step = 24; // 125 Гц
            break;
        case 2:
            step = 48; // 250 Гц
            break;
        case 3:
            step = 96; // 500 Гц
            break;
        case 4:
            step = 192; // 1000 Гц
            break;
        case 5:
            step = 24; // 125 Гц
            num_step = 1;
            break;
        }
    }
}
\end{minted}
\end{code}

% Проверить работу в содержании, правильную нумерацию приложений и переброску по ссылки.
%\chapter*{Приложение A}
%\phantomsection
%\addcontentsline{toc}{chapter}{Приложение A}
%\label{appendixA}


\makelastpage
\end{document}

