\chapter*{Введение}
\addcontentsline{toc}{chapter}{Введение}

\textbf{Актуальность}

	На сегодняшний день разрабатывается достаточно много генераторов сигналов, которые используются для различных исследований или необходимы для настройки каких-либо устройств. Существует два основных вида генераторов сигналов --- аналоговый и цифровой.
	
	Ранее практически все лабораторные генераторы были аналоговыми и конструировались на различных схемах. К их достоинствам можно отнести простоту и надёжность, но у них есть существенные недостатки в виде меньшей стабильности и более тщательной настройке. Сейчас практически все генераторы, которые есть на рынке создаются на основе цифровых методов синтеза аналоговых сигналов. Цифровые генераторы легко интегрируются с другими системами и могут управляться через программное обеспечение, что упрощает процесс. Они стабильные и точные, а также могут быть реализованы просто на микроконтроллере. Такого рода генераторы могут найти применение и в промышленности. Возможно являться компонентами в сложных схемах или помогать в настройке и тестировании оборудования. Разработанный в данной работе генератор не претендует на применение в промышленности, но в качестве простого и дешёвого функционального генератора найдёт своё применение.
	
	Применением такого генератора может быть генерация сигналов разных форм, работа с аналоговыми системами для исследования влияния сигналов на них, изучение методов обработки сигнала или основ радиоэлектроники. 
	
\textbf{Цель}
выпускной квалификационной работы состоит в создании программы для генерации сигналов на микроконтроллере STM32.

\textbf{Задачи}

\begin{enumerate}
\item Рассмотреть семейства микроконтроллеров и осуществить выбор.
\item Выбрать среду разработки.
\item Изучить методы генерации сигналов.
\item Спроектировать генератор.
\item Сконструировать макет.
\item Разработать и протестировать программу.
\end{enumerate}

