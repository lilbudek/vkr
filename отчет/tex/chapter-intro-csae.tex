\chapter*{Введение}
\addcontentsline{toc}{chapter}{Введение}

\textbf{Актуальность}

	Генератор сигналов --- это неотъемлемый инструмент для любого специалиста в области электроники. На сегодняшний день разрабатывается достаточно много генераторов сигналов, которые используются для различных исследований или необходимы для настройки каких-либо устройств, но не все генераторы, которые есть на рынке, обладают компактными размерами, лёгкостью транспортировки и доступностью в цене, а также гибкостью программного обеспечения. %Существует два основных вида генераторов --- аналоговый и цифровой.%
	
	Ранее практически все лабораторные генераторы были аналоговыми и конструировались на различных схемах. К их достоинствам можно отнести простоту и надёжность, но у них есть существенные недостатки в виде меньшей стабильности и более тщательной настройке. Сейчас практически все генераторы, которые есть на рынке создаются на основе цифровых методов синтеза аналоговых сигналов. %Цифровые генераторы легко интегрируются с другими системами и могут управляться через программное обеспечение, что упрощает процесс.%
	 Они стабильные и точные, а также могут быть реализованы просто на микроконтроллере. Такого рода генераторы могут найти применение и в промышленности, но не всем пользователям требуются такие высокие характеристики. Разработанный в данной работе генератор претендует на применение в домашней лаборатории в качестве простого и дешёвого функционального генератора сигналов.
	
	Применением такого генератора может быть генерация сигналов разных форм, работа с аналоговыми системами для исследования влияния сигналов на них, изучение методов обработки сигнала или основ радиоэлектроники. 
	
\textbf{Цель}
выпускной квалификационной работы состоит в разработке программного генератора сигналов на микроконтроллере STM32F103.

\textbf{Задачи}

\begin{enumerate}
\item Рассмотреть семейства микроконтроллеров и осуществить выбор.
\item Выбрать среду разработки.
\item Исследовать методы генерации сигналов и осуществить выбор.
\item Спроектировать генератор.
\item Сконструировать макет.
\item Разработать и протестировать программу.
\end{enumerate}

