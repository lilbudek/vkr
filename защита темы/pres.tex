\documentclass[10pt]{beamer}

\usepackage[T2A]{fontenc}
\usepackage{diagbox}
\usepackage[utf8]{inputenc}
\usepackage[russian]{babel}
\usepackage{multicol}
\usepackage{hyperref}
\usepackage{caption}
\usepackage{subcaption}
\setbeamertemplate{navigation symbols}{} %отключение значков
\setbeamertemplate{caption}[numbered]
\usetheme{Warsaw}
\setbeamertemplate{footline}{%
    \hspace{0.94\paperwidth}%
    \usebeamerfont{title in head/foot}%
    \insertframenumber\,/\,\inserttotalframenumber%
}
\newcommand{\pdiff}[2]{\frac{\partial #1}{\partial #2}}
\newcommand{\op}[1]{\mathop{\mathrm{#1}}}
\graphicspath{{pictures/}}
\begin{document}

\title{Создание программного генератора сигналов разной формы на микроконтроллере STM32.}
\author{Студент гр. 506: Вебер Д.С.\\Научный руководитель:  ст.пр. Уланов П.Н.}
\date{2023}
\institute{Алтайский государственный университет}


\frame{\titlepage}

\begin{frame}{Цель и задачи}
  \textbf{Цель работы:} разработка программы для генерации сигналов на микроконтроллере STM32.

  \textbf{Задачи:} 
  \begin{enumerate}
  \item Исследовать существующие способы программной генерации сигналов.
  \item Выбрать микроконтроллер для реализации. 
  \item Выбрать среду разработки. 
  \item Написать программу.
  \item Провести проверку работоспособности.
  \end{enumerate}
\end{frame}

\begin{frame}{Микроконтроллеры}

\begin{center}
\begin{tabular}{|l|c|c|c|c|}
\hline 
\diagbox[width=10em]{Параметр}{МК} & F103RCT6 & F302CBT6 & F373CCT6 & G431CBT6 \\ 
\hline 
Память программ & 256 Кбайт & 128 Кбайт & 256 Кбайт & 128 Кбайт \\ 
\hline 
ОЗУ & 48 Кбайт & 32 Кбайт & 32 Кбайт & 32 Кбайт \\ 
\hline 
Тактовая частота & 72 МГц & 72 МГц & 72 МГц & 170 МГц \\ 
\hline 
Кол-во вх/вых & 51 & 37 & 36 & 38 \\ 
\hline 
ЦАП & 2х12б & 1х12б & 3х12б & 4х12б \\
\hline 
\end{tabular}
\end{center} 

\end{frame}

\end{document}
